\documentclass[11pt]{beamer}
\usepackage{amsmath, amsfonts, amscd, amssymb, amsthm, tikz,pgfplots}
\setbeamertemplate{navigation symbols}{}

\usepackage{caption}

\usetheme{eaperry}


\usepackage{changepage}
\usepackage{array}
%\usepackage{multirow}
%\usepackage{float}
%\usepackage{tabu}
%\usepackage{threeparttable}
%\usepackage{threeparttablex}
%\usepackage[normalem]{ulem}
%\usepackage{makecell}
%\usepackage{xcolor}


\bibliographystyle{aer}
\usepackage{natbib}

\title{Modeling Green Development}
\subtitle{A Rosen-Roback Approach}
\institute{Spellman Program}
\author{Evan Perry}
\date{July 27, 2021}

\begin{document}

\maketitlepage

\begin{frame}{Review}

\begin{exampleblock}{\large\textbf{Research Question}}
What characteristics of urban neighborhoods relate to the number of certified green commercial buildings?
\end{exampleblock}

\vfill
Previously,
\begin{itemize}
	\item Alonso-Muth-Mills Model suggests green development occurs outside the city center
	\item Even with modifications, still problematic
	\item Need more than just distance from the city center
\end{itemize}

\end{frame}


\begin{frame}{Outline}

\begin{block}{\textbf{Today's Goal}}
Create a general equilibrium model that describes how a developer might decide where to build green. 
\end{block}

	\tableofcontents[hideallsubsections]
\end{frame}

\newsection{Model Environment}

\begin{frame}{Model Environment: Among Neighborhoods}
\small
Assume a world composed of many distinct neighborhoods

\vfill
Neighborhood Components:
\begin{itemize}
	\item Agents: Workers, Firms, One Developer
	\item Fixed Features: Housing Stock, Commercial Land, ``Public Capital"
	\item Agents and (Tradeable) Goods can move freely between and within Neighborhoods
\end{itemize}

\vfill
Key Conditions: All Agents Indifferent Between Neighborhoods (Spatial Equilibrium)

\end{frame}


\begin{frame}{Model Environment: Within the Neighborhood}

\footnotesize
\centering
\begin{tikzpicture}[scale = 2.5]
\draw[EAPyellow, fill = EAPyellow!20!, rounded corners] (0,0) rectangle (2,1.3) node[text width=4.2cm, pos=.5, black] {\textbf{Labor Market }\\ \vspace{.5em} Actors: Workers, Firms \\ \vspace{.5em} Choose: Wages, Population\\ \vspace{.5em} Need: Labor Supply, Labor Demand};
\draw[EAPgreen, fill = EAPgreen!20!, rounded corners] (2.3, 0) rectangle (4.3, 1.3) node[text width=4.2cm, pos=.5, black] {\textbf{Product Market }\\ \vspace{.5em} Actors: Firms, World \\ \vspace{.5em} Choose: Output, No. of Firms\\ \vspace{.5em} Need: Revenues, Costs};
\draw[EAPviolet, fill = EAPviolet!20!, rounded corners] (.9, -0.2) rectangle (3.4, -1.5) node[text width=5.5cm, pos=.5, black] {\textbf{Commercial Real Estate Market }\\ \vspace{.5em} Actors: Firms, Developer \\ \vspace{.5em} Choose: Price \& Quantity of Commercial RE, Green/Brown\\ \vspace{.5em} Need: Supply, Demand, Adoption\\ Condition};
\end{tikzpicture}
\vfill
Key Conditions: All Agents Optimize in Each Market
\end{frame}

\newsection{Specifying the Model}

\begin{frame}{The Worker}

$$\max_{H,X} \left\{ U(H,X) = \theta H^\alpha X^{1-\alpha}\right\} \hspace{.5cm}\text{s.t.}\hspace{.5cm} W = p_H H + X$$

\centering
\begin{tabular}{c l l}
$H$ & \alert{Housing (sq.ft.)} \\
$X$ & \alert{Composite Good} \\
$W$ & \alert{Wage} \\
$\theta$ & Amenity Index \\
$p_H$ & \alert{Price of Housing (per sq.ft.)} \\
$p_x$ & Price of Composite Good (Numeraire)
\end{tabular}

\vfill
\alert{Endogenous}, Exogenous

\end{frame}


\begin{frame}{The Firm}

$$\max_{N, Z} \left\{ \pi(N, Z:d) = A\lambda_d N^\beta Z_d^\gamma \left(\frac{\bar{K}}{M}\right)^{1-\beta-\gamma} - WN - p_{z_{d}} Z_d - \kappa \right\}$$

\centering
\begin{tabular}{c l}
$N$ & \alert{Number of Workers (Population)}\\
$Z$ & \alert{Quantity of Commercial Real Estate (sq.ft.)}\\
$\bar{K}$ & Neighborhood Capital \\
$M$ & \alert{Number of Firms} \\
$d$ & \alert{Design (Green or Brown)} \\
$A$ & Neighborhood Productivity \\
$\lambda$ & Energy Efficiency \\
$p_{z}$ & \alert{Price of Commercial Real Estate} \\
$\kappa$ & Fixed Capital Cost
\end{tabular}

\vfill
\alert{Endogenous}, Exogenous

\end{frame}


\begin{frame}{The Developer}

There is only one developer in a neighborhood, but it will still act competitively

$$\max_{h, d} \left\{  \pi^\text{Dev}(h,d:\bar{\ell}) = p_{z_d} h_d \bar{\ell} -  c_d h_d^\delta \bar{\ell} - p_\ell \bar{\ell}\right\}$$

\vfill
\centering
\begin{tabular}{c l}
$h$ & \alert{Height} \\
$\bar{\ell}$ & Commercial Land \\
$d$ & \alert{Design (Green or Brown), $d\in \{g, b\}$} \\
$c$ & Material Cost \\
$p_\ell$ & \alert{Price of Commercial Land}\\
$\delta$ & Height Friction Parameter, $\delta > 1$
\end{tabular}

\vfill
\alert{Endogenous}, Exogenous

\end{frame}


\begin{frame}{Derivation Overview}

\begin{enumerate}
\item Optimization Conditions
\begin{itemize}
	\item First-Order Conditions
	\item Market Clearing: Labor Market, Commercial Real Estate Market
\end{itemize}

\vfill
\item Spatial Equilibrium Conditions
\begin{itemize}
	\item Agents must be indifferent between where they are and anywhere else
	\item Workers -- Uniform Utility
	\item Firms and the Developer -- Zero Profits
\end{itemize}

\vfill
\item Five equations in five unknowns ($N$, $W$, $Z_d$, $p_{z_d}$, $M$)
\end{enumerate}


\end{frame}


\newsection{Results}{\textit{Conditions and Comparative Statics}}

\begin{frame}{Design}

A developer will choose the design that maximizes the price it can afford to pay for the land. This leads to the result:

\vfill
\begin{block}{The Adoption Condition}
The developer will choose to build green if and only if
$$\frac{p_{z_g}^\delta}{c_g} > \frac{p_{z_b}^\delta}{c_b}$$
where $c_g$ and $c_b$ are exogenous and $\delta >1$.
\end{block}


\end{frame}


\begin{frame}{Do Neighborhood Characteristics Matter?}

\begin{tikzpicture}
\begin{axis}[
%    title={Title},
    xlabel={Productivity Index ($A$)},
    ylabel={Adoptiong Condition Ratio},
%    xtick={0,10000,20000,30000,40000,50000},
	ytick={},
%    legend style={at={(0.5,-0.1)},
	legend pos = outer north east,
    ymajorgrids=true,
    grid style=dashed,
    % only marks,
    every axis plot/.append style={ultra thick},
    ticklabel style = {font=\scriptsize}
]

\addplot[
    color=EAPred,
    mark = square,
    mark size = 1pt
    ]
    coordinates {
   (70,0.9604758)(72,0.9797136)(74,0.9987938)(76,1.017722)(78,1.036503)(80,1.055142)(82,1.073644)(84,1.092013)(86,1.110252)(88,1.128367)(90,1.14636)(92,1.164235)(94,1.181995)(96,1.199644)(98,1.217184)(100,1.234619)(102,1.251951)(104,1.269182)(106,1.286316)(108,1.303354)(110,1.320299)(112,1.337153)(114,1.353918)(116,1.370596)(118,1.38719)(120,1.4037)(122,1.420129)(124,1.436478)(126,1.45275)(128,1.468945)(130,1.485066)
    };
\addplot[
    color=EAPgreen,
    mark = square,
    mark size = 1pt
    ]    
    coordinates{
(70,0.9830021)(72,1.002691)(74,1.022219)(76,1.041591)(78,1.060813)(80,1.079889)(82,1.098825)(84,1.117624)(86,1.136291)(88,1.154831)(90,1.173246)(92,1.19154)(94,1.209717)(96,1.22778)(98,1.245731)(100,1.263575)(102,1.281313)(104,1.298948)(106,1.316484)(108,1.333922)(110,1.351264)(112,1.368513)(114,1.385672)(116,1.402741)(118,1.419724)(120,1.436621)(122,1.453436)(124,1.470168)(126,1.486822)(128,1.503397)(130,1.519896)
    };
    \legend{\tiny Brown Ratio, \tiny Green Ratio}
\end{axis}
\end{tikzpicture}


\end{frame}


\begin{frame}{The Problematic Premium}

The premium occurs proportionally

\vfill
$$\frac{p_{z_g}}{p_{z_b}} = \left(\frac{\lambda_g}{\lambda_b}\right)^{\phi_1} \left(\frac{c_g}{c_b}\right)^{\phi_2}$$

\vfill
With the adoption condition, implies there are only two ways to change the design decision: $\lambda$ and $c$.

\end{frame}


\begin{frame}{City Characteristics Matter}

\begin{tikzpicture}
\begin{axis}[
%    title={Title},
    xlabel={Energy-Efficiency Ratio $(\lambda_g / \lambda_b)$},
    ylabel={Adoptiong Condition Ratio},
%    xtick={0,10000,20000,30000,40000,50000},
	ytick={},
%    legend style={at={(0.5,-0.1)},
	legend pos = outer north east,
    ymajorgrids=true,
    grid style=dashed,
    % only marks,
    every axis plot/.append style={ultra thick},
    ticklabel style = {font=\scriptsize}
]

\addplot[
    color=EAPred,
    mark = square,
    mark size = 1pt
    ]
    coordinates {
   (1,1.444624)(1.02,1.464904)(1.04,1.485066)(1.06,1.505114)(1.08,1.52505)(1.1,1.544877)(1.12,1.564598)(1.14,1.584215)(1.16,1.60373)(1.18,1.623146)(1.2,1.642465)
    };
\addplot[
    color=EAPgreen,
    mark = square,
    mark size = 1pt
    ]    
    coordinates{
(1,1.382559)(1.02,1.421648)(1.04,1.461051)(1.06,1.500765)(1.08,1.540786)(1.1,1.58111)(1.12,1.621734)(1.14,1.662656)(1.16,1.703871)(1.18,1.745377)(1.2,1.787171)
    };
    \legend{\tiny Brown Ratio, \tiny Green Ratio}
\end{axis}
\end{tikzpicture}

\end{frame}


\begin{frame}{Making $\lambda$ an Exponent}
\scriptsize
$$\max_{N, Z} \left\{ \pi(N, Z:d) = A^{\lambda_d} N^\beta Z_d^\gamma \left(\frac{\bar{K}}{M}\right)^{1-\beta-\gamma} - WN - p_{z_{d}} Z_d - \kappa \right\}$$

\begin{tikzpicture}
\begin{axis}[
%    title={Title},
    xlabel={Productivity ($A$)},
    ylabel={Adoptiong Condition Ratio},
%    xtick={0,10000,20000,30000,40000,50000},
	ytick={},
%    legend style={at={(0.5,-0.1)},
	legend pos = outer north east,
    ymajorgrids=true,
    grid style=dashed,
    % only marks,
    every axis plot/.append style={ultra thick},
    ticklabel style = {font=\scriptsize}
]

\addplot[
    color=EAPred,
    mark = square,
    mark size = 1pt
    ]
    coordinates {
   (70,0.6945319)(72,0.708443)(74,0.7222401)(76,0.7359273)(78,0.7495083)(80,0.7629865)(82,0.7763654)(84,0.789648)(86,0.8028373)(88,0.815936)(90,0.828947)(92,0.8418726)(94,0.8547153)(96,0.8674774)(98,0.880161)(100,0.8927682)(102,0.905301)(104,0.9177612)(106,0.9301507)(108,0.9424712)(110,0.9547243)(112,0.9669116)(114,0.9790347)(116,0.991095)(118,1.003094)(120,1.015033)(122,1.026913)(124,1.038735)(126,1.050502)(128,1.062213)(130,1.07387)
    };
\addplot[
    color=EAPgreen,
    mark = square,
    mark size = 1pt
    ]    
    coordinates{
(70,0.6836083)(72,0.6982693)(74,0.7128301)(76,0.7272941)(78,0.7416644)(80,0.7559442)(82,0.7701361)(84,0.7842428)(86,0.7982669)(88,0.8122107)(90,0.8260766)(92,0.8398666)(94,0.8535828)(96,0.8672272)(98,0.8808017)(100,0.894308)(102,0.9077477)(104,0.9211226)(106,0.9344342)(108,0.9476839)(110,0.9608732)(112,0.9740035)(114,0.9870761)(116,1.000092)(118,1.013053)(120,1.02596)(122,1.038813)(124,1.051615)(126,1.064366)(128,1.077067)(130,1.08972)
    };
    \legend{\tiny Brown Ratio, \tiny Green Ratio}
\end{axis}
\end{tikzpicture}


\end{frame}


\begin{frame}{Next Steps}

Continue to refine model:
\begin{itemize}
	\item Get clear interaction between area characteristics and the adoption decision
	\begin{itemize}
		\item Give workers preferences over working for green firms
		\item Non-proportional $\lambda$
	\end{itemize}
	\item Too deterministic and homogenous
	\item What variables are actually useful and observable? 
\end{itemize}

\bigskip
Collect data to address previous concerns, and for estimation

\end{frame}


\begin{frame}{References}
\nocite{*}
\bibliography{References}
\end{frame}


\fakesection{Appendix}{\textit{Some Additional Math}}


\begin{frame}{System of Equations}

Labor Supply and Spatial Equilibrium for the Worker
$$N W^\frac{\alpha - 1}{\alpha} = \left(\frac{\theta}{\bar{V}}\right)^\frac{1}{\alpha} \bar{H} (1-\alpha)^{1-\alpha}$$
Labor Demand (given $p_{z_d}$)
$$N\left( W^{1-\gamma} p_{z_d}^{\gamma}\right)^\frac{1}{1-\beta-\gamma} = \left( A\lambda_d \beta^{1-\gamma} \gamma^\gamma\right)^\frac{1}{1-\beta-\gamma} \bar{K} $$

Commercial Real Estate Supply
$$Z p_{z_d}^\frac{-1}{\delta - 1} = \left(\delta c_d \right)^\frac{-1}{\delta -1} \bar{\ell_c} $$

Commercial Real Estate Demand (given $W$)
$$Z\left(W^\beta p_{z_d}^{1-\beta}\right)^\frac{1}{1-\beta-\gamma} = \left( A\lambda_d \beta^\beta \gamma^{1-\beta}\right)^\frac{1}{1-\beta-\gamma} \bar{K}$$

Zero-Profit condition for the Firm (Spatial Equilibrium)
$$\left( W^\beta p_{z_d}^\gamma \right)^\frac{1}{1-\beta-\gamma} M = \left(\frac{\Phi \bar{K}}{\kappa}\right) (A \lambda_d )^\frac{1}{1-\beta-\gamma}$$

\end{frame}


\end{document}