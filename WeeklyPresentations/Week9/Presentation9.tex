\documentclass[11pt, dvipsnames, usenames]{beamer}
\usepackage{amsmath, amsfonts, amscd, amssymb, amsthm, tikz,pgfplots}
\setbeamertemplate{navigation symbols}{}

\usepackage{caption}

\usetheme{eaperry}

% \newtheorem{theorem}{Theorem}
\newtheorem{claim}[theorem]{Claim}

\usepackage{changepage}

\usepackage{sgamevar}
\renewcommand{\gamestretch}{2}
%\usepackage{array}
%\usepackage{multirow}
%\usepackage{float}
%\usepackage{tabu}
%\usepackage{threeparttable}
%\usepackage{threeparttablex}
%\usepackage[normalem]{ulem}
%\usepackage{makecell}
%\usepackage{xcolor}


\bibliographystyle{aer}
\usepackage{natbib}

\title{Modeling Green Development}
\subtitle{Another Spatial Equilibrium Model}
\institute{Spellman Program}
\author{Evan Perry}
\date{August 3, 2021}

\begin{document}

\maketitlepage

\begin{frame}{Review}

\begin{exampleblock}{\large\textbf{Research Question}}
What characteristics of urban neighborhoods relate to the number of certified green commercial buildings?
\end{exampleblock}

\vfill
Previously,
\begin{itemize}
	\item General Equilibrium / Spatial Equilibrium Model
	\item Prediction: Neighborhood characteristics do not matter
\end{itemize}

\end{frame}

\begin{frame}{Overview}

\begin{description}
\item[Purpose] How does firm location interact with green building adoption? 

\vfill
\item[Model] Describe how firms  (1) decide to buy green real estate, and (2) sort into neighborhoods

\vfill
\item[Method] Optimization and Equilibrium Characterization

\vfill
\item[Results] We find (1) adoption is unaffected by neighborhood choice, (2) firms with higher agglomeration economies move to more productive neighborhoods
\end{description}

\end{frame}

\newsection{Model Overview \& Environment}

\begin{frame}{Model Overview}

\begin{itemize}
\item Two Agents: Firms, Developers
\vfill
\item Labor Market is Exogenous
\vfill
\item Sorting Model: Agents need to determine where to locate
\vfill
\item Adoption Model: Agents need to decide green or brown
\vfill
\item Heterogeneous Firms:
\begin{itemize}
	\item Agglomeration Economies
	\item Environmental Response
\end{itemize}
\end{itemize}

\end{frame}


\begin{frame}{Model Environment}

\begin{itemize}
\item Assume an arbitrary number of neighborhoods
\item Neighborhoods have:
\begin{itemize}
	\item Unique ``Raw Productivity": $A$
	\item $A$ is the \emph{only} unique characteristic of neighborhoods ex ante
	\item Commercially Zoned Land: $\bar{\ell}$
	\item As many workers as firms want, willing to work at a wage $\bar{W}$
\end{itemize}
\end{itemize}


\vfill
Example: $A_1 < A_2 < A_3 < A_4$

\bigskip
\centering
\begin{tikzpicture}[scale = 0.6]
\draw (0,0) rectangle (3,2) node[pos =.5]{$A_1$};
\draw (4,0) rectangle (7,2) node[pos =.5]{$A_2$};
\draw (8,0) rectangle (11,2) node[pos =.5]{$A_3$};
\draw (12,0) rectangle (15,2) node[pos =.5]{$A_4$};
\end{tikzpicture}


\end{frame}


\newsection{Agents \& Behavior}

\begin{frame}{Firm Heterogeneity: Agglomeration Economies}

\begin{quote}
\small ``Agglomeration economies exist whenever people become more productive through proximity to others. They can represent the gains that come from reducing transport costs \ldots, or the \textcolor{EAPred}{advantages associated with the free flow of ideas in dense urban environments.} Agglomeration economies are the catchall explanation for why cities can be so productive" \citep{glaeser2008cities}
\end{quote}

\vfill
Summarize agglomeration effects with $\psi_i (A)$
\begin{itemize}
\item Raw Productivity: $A$
\item $\psi_i(A)$ is (super-linearly) increasing in $A$ \citep{gaubert2018firm}
\item Agglomeration Economies: $i \in \{H\textcolor{gray}{igh}, L\textcolor{gray}{ow}\}$
\item $\psi_H(A) > \psi_L(A)$
\end{itemize}

\end{frame}

\begin{frame}{Firm Heterogeneity: Green Potential}

Firms differ in the benefits they can receive from going green:
%\begin{itemize}
%	\item Primary benefits: lower operating costs
%	\item Secondary benefits: higher ESG ratings, attract/retain employees  
%\end{itemize}
%
\vfill
Summarize benefits with $\lambda_{jd}$:
\begin{itemize}
	\item Potential: $j \in \{ H\textcolor{gray}{igh~Potential} , L\textcolor{gray}{ow~Potential} \}$
	\item Design: $d \in \{ g\textcolor{gray}{reen}, b\textcolor{gray}{rown} \}$
	\item $\lambda_{Hb} < \lambda_{Lb} < \lambda_{Lg} < \lambda_{Hg}$ 
\end{itemize}
\end{frame}


\begin{frame}{Firm Types}

\hspace{1cm}
\begin{game}{2}{2}[Green Potential][Agg. Economies]
            \> Low \> High \\
        Low \> $L, L$ \> $L, H$\\
        High \> $H, L$ \> $H, H$\\
\end{game}

\vfill
How does each type choose its design and neighborhood?

\end{frame}


\begin{frame}{Firm's Problem}

Choose inputs ($N\textcolor{gray}{umber~of~workers}, R\textcolor{gray}{eal~estate}$), Design ($g, b$), and neighborhood ($A$): 

\vfill
$$\max_{N, R, d, A} \left\{ \psi_i(A) \lambda_{jd} N^\beta R^\gamma \bar{K} - \bar{W}N - p_d R - k_{ij} \right\}$$

\vfill
\centering
\begin{tabular}{c l}
$\psi$ & Agglomeration effect to firm type $i$\\
$\lambda$ & Benefit from design $d$ to firm type $j$\\
$\bar{K}$ & Fixed (tradeable) capital inputs\\
$W$ & Wage\\
$p_d$ & Price per sq.ft. with design $d$\\
$k_{ij}$ & Fixed capital cost for firm type $ij$  
\end{tabular}

\end{frame}

\begin{frame}{Developer's Problem}

Chooses $h\textcolor{gray}{eight}$ and $\ell\textcolor{gray}{and}$ for both green and brown real estate, subject to its land use constraint:

\vfill
$$\max_{h_g, h_b, \ell_g, \ell_b} \left\{ \pi_g(h_g, \ell_g) + \pi_b(h_b, \ell_b)\right\} \hspace{.5cm}\text{s.t.}\hspace{.5cm} \bar{\ell} = \ell_g + \ell_b$$ 
where
$$\pi_d(h, \ell) = p_d h\ell - c_d h^\delta \ell - p_\ell \ell$$
\vfill
Assume $c_g > c_b$ and $\delta > 1$ 

\end{frame}


\newsection{Equilibrium Characterization}

\begin{frame}{The Adoption Decision}

\begin{block}{The Adoption Condition}
Green real estate is bought and built if and only if:
$$\left(\frac{c_g}{c_b}\right)^\frac{\gamma}{\delta} < \frac{\lambda_{jg}}{\lambda_{jb}}$$
\end{block}

\vfill
Where is $A$? Again, neighborhood characteristics do not affect the adoption decision!

\end{frame}


\begin{frame}{Firm Sorting: Agglomeration Economies}

\begin{claim}
% For firms 1 and 2 with all else equal, if $\sigma_1 < \sigma_2$, then $A_1^* \leq A_2^*$ in spatial equilibrium.
In our example, type High Agglomeration Economy firms will locate in neighborhoods with at least as high $A$ as any Low Agglomeration Economy Firm.
\end{claim}

\vfill
\begin{itemize}
\item Type $H$ firms can make better use of high $A$ neighborhoods
\item The increase production more than offsets the increase in real estate prices
\end{itemize}

\end{frame}

\begin{frame}{Example Sorting}

With $A_1 < A_2 < A_3 < A_4$, 
\begin{center}
\begin{tikzpicture}[scale = .7]
\draw[fill =EAPred!40!] (0,0) rectangle (2,1) node[pos=.5]{$L$};
\draw[fill=EAPblue!40!] (0,1) rectangle (2,2) node[pos=.5]{$L$};
\draw[fill = brown!80!] (0,2) rectangle (2,3) node[pos=.5]{$b$};
\draw[fill =EAPred!40!] (6,0) rectangle (8,1) node[pos=.5]{$L$};
\draw[fill =EAPblue!85!] (6,1) rectangle (8,2) node[pos=.5]{$H$};
\draw[fill = brown!80!] (6,2) rectangle (8,3) node[pos=.5]{$b$};
\draw[fill =EAPred!85!] (3,0) rectangle (5,1) node[pos=.5]{$H$};
\draw[fill =EAPblue!40!] (3,1) rectangle (5,2) node[pos=.5]{$L$};
\draw[fill = ForestGreen] (3,2) rectangle (5,3) node[pos=.5]{$g$};
\draw[fill =EAPred!85!] (9,0) rectangle (11,1) node[pos=.5]{$H$};
\draw[fill =EAPblue!85!] (9,1) rectangle (11,2) node[pos=.5]{$H$};
\draw[fill = ForestGreen] (9,2) rectangle (11,3) node[pos=.5]{$g$};
\draw (-.3,-.3) rectangle (2.3, 3.3);
\node[below] at (1, -.3) {$A_1$};
\draw (2.7,-.3) rectangle (5.3, 3.3);
\node[below] at (4, -.3) {$A_2$};
\draw (5.7,-.3) rectangle (8.3, 3.3);
\node[below] at (7, -.3) {$A_3$};
\draw (8.7,-.3) rectangle (11.3, 3.3);
\node[below] at (10, -.3) {$A_4$};
\node[left] at (-1, .5) {Green Potential};
\node[left] at (-1, 1.5) {Agg. Economies};
\node[left] at (-1, 2.5) {Design};
\end{tikzpicture}
\end{center}

The ordering within Agg. Economies is arbitrary

\end{frame}

\begin{frame}{How Are Types Distributed?}

\centering
\begin{columns}
\begin{column}{.4\textwidth}
\textbf{High $\psi_i$ Industries:}
\vspace{.5cm}

\begin{itemize}
\item Finance
\vspace{.25cm}

\item Technology
\vspace{.25cm}

\item Media
\end{itemize}
\vspace{.75cm}
From \cite{glaeser2008cities}

\end{column}
\begin{column}{.49\textwidth}
\textbf{High Environmental-Social Governance Industries:}
\vspace{.5cm}

\begin{itemize}
\item Technology (10)
\vspace{.25cm}

\item Medical Supply/Research (8)

\vspace{.25cm}
\item Finance \& Insurance (8)

\vspace{.25cm}
\item Utilities \& Energy (5)
\end{itemize}

\vspace{.5cm}
From \href{https://www.investors.com/research/best-esg-companies-top-stocks-environmental-social-governance-values/}{MSCI ESG Research}
\end{column}
\end{columns}

\end{frame}

\begin{frame}{Next Week}

Continue to build the model with a focus on generalizing the model, narrowing in on an equilibrium, and working towards a testable model.

\vfill
Consider:
\begin{itemize}
\item Arbitrary Sectors and Continuous $\lambda$
\item Prove Existence and Clarify Equilibrium
\item Link Agglomeration and Environmentalism -- Reputation
\item How would we estimate this model?
\end{itemize}

\end{frame}


\begin{frame}{References}
\nocite{*}
\bibliography{References}
\end{frame}

%\begin{frame}{Proof Sketch: Claim 1}
%
%\end{frame}

\end{document}