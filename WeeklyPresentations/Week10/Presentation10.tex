\documentclass[11pt, dvipsnames, usenames]{beamer}
\usepackage{amsmath, amsfonts, amscd, amssymb, amsthm, tikz,pgfplots}
\setbeamertemplate{navigation symbols}{}

\usepackage{caption}

\usetheme{eaperry}

% \newtheorem{theorem}{Theorem}
\newtheorem{claim}[theorem]{Claim}

\usepackage{changepage}

\usepackage{sgamevar}
\renewcommand{\gamestretch}{2}
%\usepackage{array}
%\usepackage{multirow}
%\usepackage{float}
%\usepackage{tabu}
%\usepackage{threeparttable}
%\usepackage{threeparttablex}
%\usepackage[normalem]{ulem}
%\usepackage{makecell}
%\usepackage{xcolor}


\bibliographystyle{aer}
\usepackage{natbib}

\title{Green Development within\\ Urban Environments}
\institute{Spellman Program}
\author{Evan Perry}
\date{August 10, 2021}

\begin{document}

\maketitlepage

\begin{frame}{Review}

\begin{exampleblock}{\large\textbf{Research Question}}
What characteristics of urban neighborhoods relate to the number of certified green commercial buildings?
\end{exampleblock}

\vfill
Previously,
\begin{itemize}
	\item Spatial equilibrium model of firm sorting and green building adoption
	\item Missing: Clear equilibrium, specific prediction, testable prediction
\end{itemize}

\end{frame}

\begin{frame}{Overview}

\begin{block}{Today's Goal}
Continue to investigate last week's model by clarifying its predictions and beginning to consider it empirically.
\end{block}

\tableofcontents

\end{frame}

\newsection{Model Review \& Refinement}

\begin{frame}{Model Environment \& Overview}

$\mathcal{N}$, a set of neighborhoods:
\begin{itemize}
	\item Fixed number of workers $N$
	\item Ex ante, $N$ is the only difference between neighborhoods
	\item Fixed amount of commercially developable land
\end{itemize}

\vfill
Sorting Model: Where do different firms locate?

\vfill
Adoption Model: Which firms go green?

\end{frame}

\begin{frame}{Agents}

Firms:
\begin{itemize}
	\item Choose inputs (labor, real estate), design of real estate (green/brown), and neighborhood ($N$)
	\item Differ by:
	\begin{itemize}
		\item Agglomeration Economies (sector)
		\item Green Benefits (individual firms)
	\end{itemize}
\end{itemize}

\vfill
Developer:
\begin{itemize}
	\item Chooses height, land footprint for green and brown buildings
	\item Higher material costs for green construction
\end{itemize}

\end{frame}


\begin{frame}{Sorting}

\begin{enumerate}
	\item Firms with higher Agglomeration Economies locate in larger (higher $N$) neighborhoods
	
	\vfill
	\item Firms do not sort based on their Green Benefits
	
	\vfill
	\item Within neighborhoods:
	\begin{itemize}
		\item Agglomeration Economies are homogeneous
		\item Firms differ only in their Green Benefits ($H\textcolor{gray}{igh}$, $L\textcolor{gray}{ow}$)
	\end{itemize}
\end{enumerate}

\end{frame}

\begin{frame}{Green Adoption}

Differences in the proportion of Green real estate between neighborhoods come from the distribution of firm types:

\vfill
\textbf{Scenario \#1}: All Firms are equally likely to have a high Green Benefit

\vfill
\textbf{Scenario \#2}: Higher Agglomeration Economy Firms are more likely to have a high Green Benefit
\end{frame}


\newsection{Simulation}

\begin{frame}{Price of Commercial Real Estate}

\begin{tikzpicture}
\begin{axis}[
%    title={Title},
    xlabel={No. of Workers ($N$), Thousands},
    ylabel={Price per Sq. ft.},
%    xtick={0,10000,20000,30000,40000,50000},
	ytick={},
%    legend style={at={(0.5,-0.1)},
	legend pos = outer north east,
    ymajorgrids=true,
    grid style=dashed,
    % only marks,
    every axis plot/.append style={ultra thick},
    ticklabel style = {font=\scriptsize}
]

\addplot[
    color=EAPred,
    mark = square,
    mark size = 1pt
    ]
    coordinates {
   (0.5,7.830622)(1,11.12225)(1.5,13.65648)(2,15.79753)(2.5,17.68682)(3,19.39703)(3.5,20.97139)(4,22.43807)(4.5,23.8167)(5,25.12154)(5.5,26.36339)(6,27.55064)(6.5,28.68997)(7,29.78678)(7.5,30.84554)(8,31.86999)(8.5,32.86329)(9,33.82813)(9.5,34.76684)(10,35.68146)
    };
\addplot[
    color=EAPgreen,
    mark = square,
    mark size = 1pt
    ]    
    coordinates{
(0.5,8.230062)(1,11.6896)(1.5,14.3531)(2,16.60336)(2.5,18.58903)(3,20.38648)(3.5,22.04114)(4,23.58264)(4.5,25.03159)(5,26.40299)(5.5,27.70819)(6,28.956)(6.5,30.15344)(7,31.30621)(7.5,32.41898)(8,33.49568)(8.5,34.53965)(9,35.5537)(9.5,36.5403)(10,37.50158)
    };
    \legend{\tiny Brown, \tiny Green}
\end{axis}
\end{tikzpicture}

\end{frame}


\begin{frame}{Commercial Space}

\begin{tikzpicture}
\begin{axis}[
%    title={Title},
    xlabel={No. of Workers ($N$), Thousands},
    ylabel={Commercial Real Estate, Sq. ft.},
%    xtick={0,10000,20000,30000,40000,50000},
	ytick={},
%    legend style={at={(0.5,-0.1)},
	legend pos = outer north east,
    ymajorgrids=true,
    grid style=dashed,
    % only marks,
    every axis plot/.append style={ultra thick},
    ticklabel style = {font=\scriptsize}
]

\addplot[
    color=EAPred,
    mark = square,
    mark size = 1pt
    ]
    coordinates {
   (0.5,42498.69)(1,172966.2)(1.5,393139)(2,703958.5)(2.5,1106088)(3,1600044)(3.5,2186246)(4,2865054)(4.5,3636777)(5,4501690)(5.5,5460039)(6,6512048)(6.5,7657921)(7,8897845)(7.5,10231997)(8,11660537)(8.5,13183619)(9,14801387)(9.5,16513975)(10,18321513)
    };
\addplot[
    color=EAPgreen,
    mark = square,
    mark size = 1pt
    ]    
    coordinates{
(0.5,6023.984)(1,24517.13)(1.5,55725.55)(2,99782.72)(2.5,156782.7)(3,226798.5)(3.5,309889.9)(4,406107.5)(4.5,515495.5)(5,638092.8)(5.5,773934.2)(6,923051.3)(6.5,1085473)(7,1261227)(7.5,1450336)(8,1652825)(8.5,1868714)(9,2098025)(9.5,2340776)(10,2596986)
    };
    \legend{\tiny Brown, \tiny Green}
\end{axis}
\end{tikzpicture}

\end{frame}


\begin{frame}{Firm Sorting}

\begin{tikzpicture}
\begin{axis}[
%    title={Title},
    xlabel={No. of Workers ($N$), Thousands},
    ylabel={Price of Brown Com. Space},
%    xtick={0,10000,20000,30000,40000,50000},
	ytick={},
%    legend style={at={(0.5,-0.1)},
	legend pos = outer north east,
    ymajorgrids=true,
    grid style=dashed,
    % only marks,
    every axis plot/.append style={ultra thick},
    ticklabel style = {font=\scriptsize}
]

\addplot[
    color=EAPred,
    mark = square,
    mark size = 1pt
    ]
    coordinates {
   (0.5,7.784394)(1,11.08057)(1.5,13.62255)(2,15.77245)(2.5,17.67106)(3,19.39079)(3.5,20.97474)(4,22.45103)(4.5,23.83922)(5,25.15358)(5.5,26.40488)(6,27.60151)(6.5,28.75012)(7,29.85615)(7.5,30.92405)(8,31.95755)(8.5,32.95981)(9,33.93355)(9.5,34.88108)(10,35.80445)
    };
\addplot[
    color=EAPgreen,
    mark = square,
    mark size = 1pt
    ]    
    coordinates{
(0.5,6.704508)(1,10.00921)(1.5,12.65327)(2,14.94283)(2.5,17.00039)(3,18.89016)(3.5,20.65091)(4,22.30826)(4.5,23.88022)(5,25.38002)(5.5,26.81773)(6,28.20126)(6.5,29.53693)(7,30.8299)(7.5,32.08445)(8,33.30418)(8.5,34.49214)(9,35.65097)(9.5,36.78293)(10,37.89002)
    };
    \legend{\tiny Low Agg. Effect, \tiny High Agg. Effect}
\end{axis}
\draw[dashed, thick] (3.1,0) -- (3.1,5.7);
\end{tikzpicture}

\end{frame}

\begin{frame}{Scenario \#1}

\begin{tikzpicture}
\begin{axis}[
%    title={Title},
    xlabel={No. of Workers ($N$), Thousands},
    ylabel={Prop. of Commercial Space},
%    xtick={0,10000,20000,30000,40000,50000},
	ytick={},
%    legend style={at={(0.5,-0.1)},
	legend pos = outer north east,
    ymajorgrids=true,
    grid style=dashed,
    % only marks,
    every axis plot/.append style={ultra thick},
    ticklabel style = {font=\scriptsize}
]

\addplot[
    color=EAPred,
    mark = square,
    mark size = 1pt
    ]
    coordinates {
   (0.5,0.8758522)(1,0.8758522)(1.5,0.8758522)(2,0.8758522)(2.5,0.8758522)(3,0.8758522)(3.5,0.8758522)(4,0.8758522)(4.5,0.8758522)(5,0.8758522)(5.5,0.8758522)(6,0.8758522)(6.5,0.8758522)(7,0.8758522)(7.5,0.8758522)(8,0.8758522)(8.5,0.8758522)(9,0.8758522)(9.5,0.8758522)(10,0.8758522)
    };
\addplot[
    color=EAPgreen,
    mark = square,
    mark size = 1pt
    ]    
    coordinates{
(0.5,0.1241478)(1,0.1241478)(1.5,0.1241478)(2,0.1241478)(2.5,0.1241478)(3,0.1241478)(3.5,0.1241478)(4,0.1241478)(4.5,0.1241478)(5,0.1241478)(5.5,0.1241478)(6,0.1241478)(6.5,0.1241478)(7,0.1241478)(7.5,0.1241478)(8,0.1241478)(8.5,0.1241478)(9,0.1241478)(9.5,0.1241478)(10,0.1241478)
    };
    \legend{\tiny Brown, \tiny Green}
\end{axis}
\draw[dashed, thick] (3.1,0) -- (3.1,5.7);
\end{tikzpicture}

\end{frame}

\begin{frame}{Scenario \#2}

\begin{tikzpicture}
\begin{axis}[
%    title={Title},
    xlabel={No. of Workers ($N$), Thousands},
    ylabel={Price of Brown Com. Space},
%    xtick={0,10000,20000,30000,40000,50000},
	ytick={},
%    legend style={at={(0.5,-0.1)},
	legend pos = outer north east,
    ymajorgrids=true,
    grid style=dashed,
    % only marks,
    every axis plot/.append style={ultra thick},
    ticklabel style = {font=\scriptsize}
]

\addplot[
    color=EAPred,
    mark = square,
    mark size = 1pt
    ]
    coordinates {
   (0.5,0.8758522)(1,0.8758522)(1.5,0.8758522)(2,0.8758522)(2.5,0.8758522)(3,0.8758522)(3.5,0.8758522)(4,0.8758522)(4.5,0.8758522)
    };
\addplot[
    color=EAPred,
    mark = square,
    mark size = 1pt
    ]
    coordinates {
(5,0.7232374)(5.5,0.7232374)(6,0.7232374)(6.5,0.7232374)(7,0.7232374)(7.5,0.7232374)(8,0.7232374)(8.5,0.7232374)(9,0.7232374)(9.5,0.7232374)(10,0.7232374)
    };
\addplot[
    color=EAPgreen,
    mark = square,
    mark size = 1pt
    ]    
    coordinates{
(0.5,0.1241478)(1,0.1241478)(1.5,0.1241478)(2,0.1241478)(2.5,0.1241478)(3,0.1241478)(3.5,0.1241478)(4,0.1241478)(4.5,0.1241478)
    };
\addplot[
    color=EAPgreen,
    mark = square,
    mark size = 1pt
    ]    
    coordinates{
(5,0.2767626)(5.5,0.2767626)(6,0.2767626)(6.5,0.2767626)(7,0.2767626)(7.5,0.2767626)(8,0.2767626)(8.5,0.2767626)(9,0.2767626)(9.5,0.2767626)(10,0.2767626)
    };
    \legend{\tiny Brown, ,\tiny Green}
\end{axis}
\draw[dashed, thick] (3.1,0) -- (3.1,5.7);
\end{tikzpicture}

\end{frame}


\newsection{Towards Empirical Work}


\begin{frame}{A Regression?}

If sector $i$ inhabits a neighborhood, then

\begin{align*}
\log(Rg) = \beta_0 &+ \beta_1\log(N) + \beta_2 \log(\Psi_i(N)) + \beta_3\log(\alpha_1\mu_i + \alpha_2(1 - \mu_i) )\\ &+ \beta_4\log\left( \alpha_3\left(\frac{1 - \mu_i}{\mu_i}\right) + 1\right)
\end{align*}

\centering
\vfill
\begin{tabular}{c l}
$R_g$ & Quantity of Green Real Estate\\
$N$ & No. of Workers\\
$\Psi_i$ & Agglomeration Effect for Industry $i$\\
$\mu_i$ & Proportion of High Green Benefit Firms in $i$\\
$\alpha, \beta$'s & Constants
\end{tabular}

\end{frame}


\begin{frame}{Continued Work}

Model Modifications and Extensions:
\begin{itemize}
	\item Economies of Scale in Green Construction
	\item Dynamic Model
	\item Connect Agglomeration and Green Benefits
\end{itemize}

\vfill
Empirically:
\begin{itemize}
	\item Move towards variables that are measurable and known
	\item Methodologies for estimating unobserved costs/benefits
\end{itemize}

\end{frame}


\begin{frame}{References}
\nocite{*}
\bibliography{References}
\end{frame}

\fakesection{Appendix}

\begin{frame}{Firm Problem}

Choose inputs ($N\textcolor{gray}{umber~of~workers}, R\textcolor{gray}{eal~estate}$), Design ($g, b$), and neighborhood ($N$): 

\vfill
$$\max_{L, R, d, N} \left\{ \Psi_i(N) \lambda_{jd} L^\beta R^\gamma \bar{K} - \bar{W}N - p_d R - k_{i} \right\}$$

\vfill
\centering
\begin{tabular}{c l}
$\Psi_i$ & Agglomeration effect to firm type $i$\\
$\lambda$ & Benefit from design $d$ to firm type $j$\\
$\bar{K}$ & Fixed (tradeable) capital inputs\\
$W$ & Wage\\
$p_d$ & Price per sq.ft. with design $d$\\
$k_{i}$ & Fixed capital cost for firm type $i$  
\end{tabular}

\end{frame}


\begin{frame}{Developer Problem}

Chooses $h\textcolor{gray}{eight}$ and $\ell\textcolor{gray}{and}$ for both green and brown real estate, subject to its land use constraint:

\vfill
$$\max_{h_g, h_b, \ell_g, \ell_b} \left\{ \pi_g(h_g, \ell_g) + \pi_b(h_b, \ell_b)\right\} \hspace{.5cm}\text{s.t.}\hspace{.5cm} \bar{\ell} = \ell_g + \ell_b$$ 
where
$$\pi_d(h, \ell) = p_d h\ell - c_d h^\delta \ell - p_\ell \ell$$
\vfill
Assume $c_g > c_b$ and $\delta > 1$ 

\end{frame}


\begin{frame}{Equilibrium Conditions}

\begin{enumerate}
	\item Labor Market Clearing
	\item Green Real Estate Market Clearing
	\item Brown Real Estate Market Clearing
	\item Spatial Equilibrium Condition for Green Firms
	\item Spatial Equilibrium Condition for Brown Firms
\end{enumerate}

\vfill
We derive these from the agents' problems and then proceed to solve the system of equations they create

\end{frame}


\end{document}