\documentclass[11pt]{beamer}
\usepackage{amsmath, amsfonts, amscd, amssymb, amsthm, tikz}
\setbeamertemplate{navigation symbols}{}
\usepackage{natbib}

\usepackage{pgfplots}

\usetheme{eaperry}

\bibliographystyle{aer}

\title{A Theory of Investment for\\ Energy-Efficient Technologies}
\subtitle{Part I}
\author{Evan Perry}
\institute{Spellman Program}
\date{June 15, 2021}

\begin{document}


\maketitlepage


%\begin{frame}{Overview}
%\tableofcontents%[hideallsubsections]
%\end{frame}
%
%\newsection{Introduction \& Overview}{}

\begin{frame}{Review}
\begin{exampleblock}{\large\textbf{Research Question}}
What places attract energy-efficient buildings? How do neighborhood and area characteristics relate to the number of certified energy-efficient buildings?
\end{exampleblock}

\vfill

We need some theory to describe how economic agents might invest in energy-efficient technologies.

\end{frame}


\begin{frame}{Paper \& Purpose}

{\bf Hausman, Jerry~A}, ``Individual discount rates and the purchase\\ 
\quad and  utilization of energy-using durables,'' {\it The Bell Journal of}\\ 
\quad {\it Economics}, 1979, pp.~33--54.

\vfill
How do households decide
	\begin{enumerate}[(i)]
		\item what appliance to purchase?
		\item how often to use that appliance?
	\end{enumerate}

\end{frame}


\newsection{The Model}{\textit{Setup, Utilization Decision, Purchase Decision}}


%\begin{frame}{Model Overview}
%Focus is  specifically on Air Conditioners
%
%\vfill
%\begin{enumerate}
%\item Define an economic agent
%\item Solve for the optimal electricity use
%\item Add a stochastic component to the agents' appliance preferences
%\item Given the optimal energy use, consider the optimal appliance purchase
%\end{enumerate}
%
%\end{frame}
\subsection{Defining the Agent}
\begin{frame}{Defining the Agent's Preferences}

The model focuses on Air Conditioners specifically.

\begin{equation}
U(x , z(\tau)) =  x - \eta\left(\frac{a}{2}\right)(z(\tau))^2
\end{equation}

\footnotesize
\begin{itemize}
\item $x$ : Composite Good
\item $z(\tau)$ : Degree-hours of discomfort at thermostat setting $\tau$
\item $\eta$ : Proportion of time at home
\item $a$ : Constant 
\end{itemize}


\end{frame}

\begin{frame}{Defining the Agent's Budget}

\begin{equation}
 y =  \underbrace{p \left(\frac{\eta \lambda H(\tau )}{EER}\right)}_\text{Operating Cost} + \underbrace{\psi \rho}_\text{AC Cost} + \underbrace{x}_{\substack{\text{Non-AC} \\ \text{Spending}}}
\end{equation}

\footnotesize
\begin{itemize}
\item $y$ : Income
\item $p$ : Price of Electricity
\item $KWH = \frac{\eta \lambda H(\tau )}{EER}$ : ``Quantity" of Electricity
\begin{itemize} \footnotesize
	\item $\lambda$ : ``Size" of the AC
	\item $H(\tau)$ : Degree-Hours Operating
	\item $EER$ : Energy Efficiency Rating
\end{itemize}
\item $\psi$ : Discount/Durability Factor
\item $\rho$ : Price of the AC
\end{itemize}


\end{frame}


\begin{frame}{Defining the Agent's Objective}

\[
\max U(x , z(\tau)) \hspace{1cm} \text{s.t.} \hspace{1cm} y =  p \left(\frac{\eta \lambda H(\tau )}{EER}\right) + \psi \rho + x
\]

\vfill
Or better yet,
\begin{equation}
\max U = y - p \left(\frac{\eta \lambda H(\tau )}{EER}\right) - \psi \rho - \eta\left(\frac{a}{2}\right)(z(\tau))^2
\end{equation}

\end{frame}

\subsection{Utilization Decision}
\begin{frame}{Figure 1: Utilization Decision}

\begin{tikzpicture}
\begin{axis}[
	axis lines = left,
	ticks=none,
    xlabel = {Discomfort, \( z(\tau) \)},
    ylabel = {Composite Good, $x$},
    xmin=0, xmax=27500,
	ymin=45000, ymax=70000,
	every axis plot/.append style={ultra thick}
]
\addplot[
	domain = 1:25000,
	samples=200,
	color=EAPred
]{51431.25+ 0.5*(0.0001/2)*x^2};
\addplot[
	domain = 1:25000,
	samples=200,
	color=EAPblue
]{65000 - 0.45*3000 - (0.5*0.133*80000)/8000 * (25000 - x)};
\end{axis}
\node[EAPred] at (6.75,5.125) {IC};
\node[EAPblue] at (6.75,4.3) {BC};
\node at (5, 2) {$\eta a z(\tau) = \frac{\eta p\lambda}{EER}$};
\end{tikzpicture}

\end{frame}


%\begin{frame}{Utilization Decision}
%
%Using the first-order condition,  then the utility maximizing utilization is:
%\[
%KWH = \frac{BTU}{EER}\left( \delta_1 CDH + \delta_2 \frac{p BTU}{EER}\right)
%\]
%
%%$$CDH = \int_{T_0}^\infty (t-T_0)(-\alpha \gamma)e^{-\alpha(t-T_0)}$$
%where $\delta_1 = \eta\mu$ and $\delta_2 = -\eta\mu^2/a$.
%
%\vfill
%\footnotesize
%\begin{itemize}
%\item $BTU$ : AC Capacity (``Size"), in BTUs
%\item $CDH$ : Total Cooling-Degree Hours
%\end{itemize}
%\end{frame}

\subsection{Purchase Decision}

\begin{frame}{Figure 2: High-Efficiency and Low-Efficiency ACs}

\begin{tikzpicture}
\begin{axis}[
	axis lines = left,
	ticks=none,
    xlabel = {Cooling Degree-Hours},
    ylabel = {Utility},
    xmin=0, xmax=11000,
    ymin=53500, ymax=65000,
	every axis plot/.append style={ultra thick}
]
\addplot[
	domain = 1:10000,
	samples=200,
	color=EAPred
]{65000 - 0.133*(0.5*10*(8000/8000*(x - 10*0.133*8000/10000))) - 0.45*4500 - (0.5*(0.001/2))*(0.133*80000/(0.001*10000))^2};
\addplot[
	domain = 1:10000,
	samples=200,
	color=EAPblue
]{65000 - 0.133*(0.5*10*(8000/6000*(x - 10*0.133*8000/6000))) - 0.45*1000 - (0.5*(0.001/2))*(0.133*80000/(0.001*6000))^2};
\end{axis}
\node[EAPred] at (6.75,1.2) {$U_\text{High}$};
\node[EAPblue] at (6.75,0.6) {$U_\text{Low}$};
\node at (2, 5) {\footnotesize $\Delta p KWH < \Delta \psi \rho$};
\node at (5.1, 5) {\footnotesize $\Delta p KWH > \Delta \psi \rho$};
\end{tikzpicture}

\end{frame}


\begin{frame}{Purchase Decision}

Agent $k$'s utility from appliance $i$ is:
\begin{equation}
u_{ik} = \underbrace{y + \bar{\beta}_1 p_k KWH_{ik} + \bar{\beta}_2 \rho_i + \bar{\beta}_3 (p_k BTU_k / EER_{ik})^2}_\text{Deterministic} + \underbrace{\varepsilon_{ik}}_\text{Stochastic}
\end{equation}

\vfill
The probability that appliance $i$ maximizes agent $k$'s utility is:
\begin{equation}
s_{ik} = \Pr\{ u_{ik} > u_{jk}, \text{for all}~j\neq i\}
\end{equation}

Two main factors that affect this probability: Climate and the Price of Electricity

\end{frame}

\newcommand{\y}{65000}
\newcommand{\myeta}{0.5}
\newcommand{\p}{0.133}
\newcommand{\mylambda}{80000}
\newcommand{\CDH}{25000}
\newcommand{\Htau}{12500}
\newcommand{\ztau}{12500}
\newcommand{\mypsi}{0.45}
\newcommand{\myrho}{3000}
\newcommand{\EER}{8000}
\newcommand{\mya}{0.0001}
%\newcommand{\ubar}{\pgfmathparse{\y - (\myeta * \p * \mylambda * \Htau / \EER) - (\mypsi * \myrho) - (\myeta * (\a / 2 ) * \ztau * \ztau) }}
\newcommand{\ubar}{51431.25}


\begin{frame}{Figure 3: Climate Simulation}

{\small
$$ u_{ik} = y + \bar{\beta}_1 p_k KWH_{ik} + \bar{\beta}_2 \rho_i + \bar{\beta}_3 (p_k BTU_k / EER_{ik})^2 + \varepsilon_{ik}$$
}

\begin{tikzpicture}
\begin{axis}[
%    title={Title},
    xlabel={Cooling Degree-Hours (thousands)},
    ylabel={Probability of Adoption},
%    xtick={0,10000,20000,30000,40000,50000},
%    ytick={0,20,40,60,80,100,120},
%    legend style={at={(0.5,-0.1)},
	legend pos = outer north east,
    ymajorgrids=true,
    grid style=dashed,
    % only marks,
    every axis plot/.append style={ultra thick},
    ticklabel style = {font=\scriptsize}
]

\addplot[
    color=EAPblue,
    mark = square,
    mark size = 1pt
    ]
    coordinates {
   (2,0.017)(4,0.03)(6,0.051)(8,0.08)(10,0.12)(12,0.173)(14,0.236)(16,0.31)(18,0.393)(20,0.473)(22,0.554)(24,0.632)(26,0.702)(28,0.764)(30,0.815)(32,0.86)(34,0.894)(36,0.922)(38,0.943)(40,0.959)(42,0.969)(44,0.979)(46,0.986)(48,0.99)(50,0.993)
    };
\addplot[
    color=EAPred,
    mark = square,
    mark size = 1pt
    ]    
    coordinates{
(2,0.983)(4,0.97)(6,0.949)(8,0.92)(10,0.88)(12,0.827)(14,0.764)(16,0.69)(18,0.607)(20,0.527)(22,0.446)(24,0.368)(26,0.298)(28,0.236)(30,0.185)(32,0.14)(34,0.106)(36,0.078)(38,0.057)(40,0.041)(42,0.031)(44,0.021)(46,0.014)(48,0.01)(50,0.007)
    };
    \legend{\tiny High-Efficiency AC, \tiny Low-Efficiency AC}
\end{axis}
\end{tikzpicture}


\end{frame}


\begin{frame}{Figure 4:  Electricity Price Simulation}

{\small
$$ u_{ik} = y + \bar{\beta}_1 p_k KWH_{ik} + \bar{\beta}_2 \rho_i + \bar{\beta}_3 (p_k BTU_k / EER_{ik})^2 + \varepsilon_{ik}$$
}

\begin{tikzpicture}
\begin{axis}[
%    title={Title},
    xlabel={Price of Electricity (\$/KWH)},
    ylabel={Probability of Adoption},
%    xmin=0, xmax=100,
%    ymin=0, ymax=120,
%    xtick={0,20,40,60,80,100},
%    ytick={0,20,40,60,80,100,120},
%    legend style={at={(0.5,-0.1)},
	legend pos = outer north east,
    ymajorgrids=true,
    grid style=dashed,
    every axis plot/.append style={ultra thick},
    ticklabel style = {font=\scriptsize},
    xticklabel style={
        /pgf/number format/fixed,
        /pgf/number format/precision=2
},
scaled x ticks=false
]

\addplot[
    color=EAPblue,
    mark=square,
    mark size = 1pt
    ]
    coordinates {
    (0.005,0.005)(0.015,0.008)(0.025,0.012)(0.035,0.019)(0.045,0.031)(0.055,0.048)(0.065,0.072)(0.075,0.104)(0.085,0.147)(0.095,0.2)(0.105,0.262)(0.115,0.332)(0.125,0.412)(0.135,0.491)(0.145,0.569)(0.155,0.644)(0.165,0.711)(0.175,0.77)(0.185,0.822)(0.195,0.866)(0.205,0.899)(0.215,0.926)(0.225,0.947)(0.235,0.961)(0.245,0.974)(0.255,0.982)(0.265,0.988)(0.275,0.992)
    };
\addplot[
    color=EAPred,
    mark=square,
    mark size = 1pt
    ]    
    coordinates{
    (0.005,0.995)(0.015,0.992)(0.025,0.988)(0.035,0.981)(0.045,0.969)(0.055,0.952)(0.065,0.928)(0.075,0.896)(0.085,0.853)(0.095,0.8)(0.105,0.738)(0.115,0.668)(0.125,0.588)(0.135,0.509)(0.145,0.431)(0.155,0.356)(0.165,0.289)(0.175,0.23)(0.185,0.178)(0.195,0.134)(0.205,0.101)(0.215,0.074)(0.225,0.053)(0.235,0.039)(0.245,0.026)(0.255,0.018)(0.265,0.012)(0.275,0.008)
    };
    \legend{\tiny High-Efficiency AC, \tiny Low-Efficiency AC}
\end{axis}
\end{tikzpicture}


\end{frame}

\newsection{Results}{}

%\begin{frame}{Table 1: Estimating Appliance Purchase}
%
%\footnotesize\centering
%\begin{tabular}{r  c}
%\hline\hline
%& Max. Likelihood Estimate\\
%\hline \\ [-2ex]
%$\bar{\beta}_1$ & $-$0.194\\
%& (0.110)\\ \\[-1.8ex]
%$\bar{\beta}_2$ & $-$0.0449\\
%& (0.0170)\\ \\[-1.8ex]
%$\bar{\beta}_3$ & $-$0.0151\\
%& (0.0127)\\ \\[-1.8ex]
%$\sigma_{\beta_1}$ & 0.0183\\
%& (0.0065)\\ \\[-1.8ex]
%$\sigma_{\beta_2}$ & 0.0167\\
%& (0.0088)\\ \\[-1.8ex]
%$\sigma_{\beta_3}$ & 0.124\\
%& (0.202)\\ \\[-1.8ex]
%% $L(\theta)$ & $-$46.58\\
%\hline\hline
%\multicolumn{2}{l}{No. of Observations = 46}
%\end{tabular}
%\bigskip
%
%\normalsize
%Adapted from \citep{hausman1979individual}.
%%\begin{column}{0.5\textwidth}
%%
%%\vspace{3cm}
%%$$L(\theta) = \sum_{k=1}^{N} \sum_{i=1}^{3} y_{ik} \log s_{ik}$$
%%where $\theta$ is a vector with $\bar{\beta}_1$ , $\bar{\beta}_2$, $\bar{\beta}_3$,   $\sigma_{\beta_1}$, $\sigma_{\beta_2}$, and $\sigma_{\beta_3}$.
%%\end{column}
%%\end{columns}
%\end{frame}


\begin{frame}{Table 1: Implied Discount Rates by Income}

\begin{center}
\begin{tabular}{r c c c}
\hline\hline
\multicolumn{1}{c}{Income} & Observations & $\beta_2$ & Discount Rate \\
\hline \\ [-0.8ex]
$< \$6000$ & 6 & $-0.118$ & 89\% \\ [1.2ex]
$\$6000$ -- $\$10000$ & 15 & $-0.075$ & 39\% \\ [1.2ex]
$\$10000$ -- $\$15000$ & 16 & $-0.061$ & 27\% \\ [1.2ex]
$\$15000$ -- $\$25000$ & 17 & $-0.049$ & 17\% \\ [1.2ex]
$\$25000$ -- $\$35000$ & 8 & $-0.039$ & 8.9\% \\ [1.2ex]
$\$35000$ -- $\$50000$ & 3 & $-0.031$ & 5.1\% \\ [1.2ex]
\hline\hline
\end{tabular}
\end{center}
Adapted from \citep{hausman1979individual}

\end{frame}

\begin{frame}{Contribution \& Summary}

\begin{itemize}
	\item This paper is the first major paper to suggest that the under-utilization of energy-efficient technologies is more than just an externality problem
$$\text{High}~ r \Rightarrow \text{High}~ \beta_2 \Rightarrow \text{Emphasize Purchase Price}$$
	\vfill	
	\item Leads to a wider literature exploring this apparent gap
	\vfill
	\item Provides an interesting model for the adoption of EE technology
\end{itemize}

\end{frame}


\begin{frame}{Next Week}

\begin{itemize}
\item Incorporate Hausman's random utility model into an urban sorting model
\vfill
\item {\bf Allcott, Hunt and Michael Greenstone}, ``Is there an energy\\ 
\quad efficiency gap?,'' {\it Journal of Economic Perspectives}, 2012,\\
\quad {\it 26} (1), 3--28.
\end{itemize}

\end{frame}


%\begin{frame}{Summary}
%
%\begin{itemize}
%\item Goal is to build a model with a theory of investment into energy-efficient technologies
%\item Built a discrete-choice, random-utility model with three important takeaways related to:
%\begin{itemize}
%	\item Climate
%	\item Electricity Prices
%	\item Income
%\end{itemize}
%\end{itemize}
%
%\end{frame}

\begin{frame}{References}
\nocite{*}
\bibliography{Sources}
\end{frame}


%\fakesection{Appendix}{\textit{Other Relevant Material}}
%
%
%\begin{frame}{Figure 4: Incorporating the Alonso-Muth-Mills Model}
%
%\centering
%\begin{tikzpicture}[scale = .6]
%	\draw[thick, <->] (10,0) node[below]{$x$} -- (0,0) -- (0,10) node[left]{$p_L$};
%	\draw[EAPred, ultra thick] [domain = 0:9] plot (\x, {.7*(.3/1)^(.7/.3)*((5 - .5*\x)/1)^(1/.3)});
%	\node[above, EAPred] at (9,0) {$p_L^{P}(x)$};
%	\draw[EAPblue, ultra thick] [domain = 0:9] plot (\x, {.75*(.25/1)^(.75/.25)*((10 - .3*\x - 1)/2)^(1/.25) + 0.5}) node[EAPblue, right]{$p_L^R(x)$};
%\end{tikzpicture}
%
%\end{frame}

\end{document}